\documentclass[a4paper,11pt]{article}
\usepackage[english]{babel}
\usepackage[T1]{fontenc}
\usepackage[utf8]{inputenc}
\usepackage{fancyhdr}
\usepackage{graphicx}
\usepackage{a4wide}
\usepackage{numprint}
\usepackage{url}
\usepackage{moreverb}
\usepackage{algpseudocode}
\usepackage{algorithm}
\usepackage{amsmath}
\usepackage{amssymb}
\usepackage{stmaryrd}
\usepackage{cite}
\usepackage{wasysym}
\usepackage{bussproofs}
\usepackage{pdflscape}
\usepackage{changepage}

\pagestyle{fancy}
\def\mathlarge#1{\mbox{\LARGE $#1$}}

\newcommand{\tab}{\hspace*{2em}}
\title{Lab 2: Abstract Interpretation}
\author{Andreas Sjöberg \\ \url{ansjob@kth.se} 
		\and Marcus Larsson \\ \url{marcular@kth.se}
	}

\fancypagestyle{plain}
{
	\fancyhf{}
	\renewcommand{\headrulewidth}{0pt}
	\fancyfoot[C]{}
}

\begin{document}

\thispagestyle{plain}
\maketitle
\clearpage

\tableofcontents
\clearpage

\section{Introduction}
In this lab we extend the previously developed abstract machine to execute in abstract domains.
We focus specifically on the detection of signs domain in this report.
With this extension programs may be analysed, which can be the basis for optimizations or bug detection (for example detection of dead code or uncaught exceptions).

\section{Modifications}

\subsection{Operational semantics for execution in abstract domain}
The rules of operational semantics for While have been modified to support the detection of signs analysis.
Modified rules are described below.
One should note that the language is no longer deterministic because of the branch and division instructions.

\begin{table}[h!]
\centering
\begin{tabular}{|llll|}
\hline

$\langle \texttt{PUSH-}n:c, e, ps \rangle$ & $\rhd$ & $\langle c, \textbf{abs}_{Z_{\bot}}(n):e, ps \rangle$ &\\
$\langle \texttt{ADD}:c, v_1:v_2:e, ps \rangle$ & $\rhd$ & $\langle c, (v_1 +_{SE} v_2):e, ps \rangle$ & \\
$\langle \texttt{SUB}:c, v_1:v_2:e, ps \rangle$ & $\rhd$ & $\langle c, (v_1 -_{SE} v_2):e, ps \rangle$ & \\
$\langle \texttt{MULT}:c, v_1:v_2:e, ps \rangle$ & $\rhd$ & $\langle c, (v_1 \star_{SE} v_2):e, ps \rangle$ & \\
$\langle \texttt{DIV}:c, v_1:v_2:e, ps \rangle$ & $\rhd$ &
	$\left\{
	\parbox{5cm}{
		$\langle c, (v_1 /_{SE} (v_2 \backslash \texttt{ZERO})):e, ps \rangle$ \\
 		$\langle c, (v_1 /_{SE} v_2):e, \hat{ps} \rangle $
	} \right. $
	&
	$\parbox{4cm}{
		$\text{if }  v_2 \neq \texttt{ZERO} $ \\
		$\text{if } \texttt{ZERO} \sqsubseteq_{SE} v_2$
	}$\\
$\langle \texttt{TRUE}:c, e, ps \rangle $ & $ \rhd $ & $ \langle c, \texttt{TT}:e, ps \rangle  $ &\\
$\langle \texttt{FALSE}:c, e, ps \rangle $ & $ \rhd $ & $ \langle c, \texttt{FF}:e, ps \rangle  $ &\\

$\langle \texttt{EQ}:c, v_1:v_2:e, ps \rangle $ & $ \rhd $ & $ \langle c, (v_1 =_{SE} v_2):e, ps \rangle $ &\\
$\langle \texttt{LE}:c, v_1:v_2:e, ps \rangle $ & $ \rhd $ & $ \langle c, (v_1 \leq_{SE} v_2):e, ps \rangle $ &\\
$\langle \texttt{AND}:c, v_1:v_2:e, ps \rangle $ & $ \rhd $ & $ \langle c, (v_1 \wedge_{SE} v_2):e, ps \rangle $ &\\
$\langle \texttt{NEG}:c, v:e, ps \rangle $ & $ \rhd $ & $ \langle c, (\neg_{SE} v):e, ps \rangle $ &\\
$\langle \texttt{FETCH-}x:c, e, ps \rangle $ & $ \rhd $ & $ \langle c, ps(x):e, ps \rangle $ &\\
$\langle \texttt{STORE-}x:c, v:e, ps \rangle $ & $ \rhd $ & $ \langle c, e, ps[x \mapsto v] \rangle $ &\\
$\langle \texttt{NOOP}:c, e, ps \rangle $ & $ \rhd $ & $ \langle c, e, ps \rangle $ &\\
$\langle \texttt{BRANCH}(c_1, c_2):c, v:e, ps \rangle $ & $ \rhd $ &  
	$\left\{
	\parbox{3cm}{
		$\langle c_1:c, e, ps \rangle$ \\
		$\langle c_2:c, e, ps \rangle$ \\
		$\langle c, e, \hat{ps} \rangle$ 
	} \right.$
	&
	\parbox{3cm}{
		if $ \texttt{TT} \sqsubseteq_{TE} v$ \\
		if $ \texttt{FF} \sqsubseteq_{TE} v$\\
		if $ \texttt{ERR}_B \sqsubseteq_{TE} v $
	} \\
$\langle \texttt{LOOP}(c_1, c_2):c, e, ps \rangle $ & $ \rhd $ & \multicolumn{2}{l|}{$\langle c_1:\texttt{BRANCH}(c_2:\texttt{LOOP}(c_1, c_2), \texttt{NOOP}):c, e, ps \rangle $ }\\
$\langle \texttt{STORE-}x:c, e, \hat{ps} \rangle $ & $ \rhd $ & $ \langle c, e, \hat{ps} \rangle $ &\\
$\langle \texttt{BRANCH}(c_1, c_2):c, v:e, \hat{ps} \rangle $ & $ \rhd $ & $ \langle c, e, \hat{ps} \rangle $ &\\
$\langle \texttt{LOOP}(c_1, c_2):c, e, \hat{ps} \rangle $ & $ \rhd $ & $ \langle c, e, \hat{ps} \rangle $ &\\
$\langle \texttt{TRY}(c_1):\texttt{CATCH}(c_2):c, e, \hat{ps} \rangle $ & $ \rhd $ & $ \langle c, e, \hat{ps} \rangle $ &\\
$\langle \texttt{TRY}(c_1):c, e, ps \rangle $ & $ \rhd $ & $ \langle c_1:c, e, ps \rangle $ &\\
$\langle \texttt{CATCH}(c_1):c, e, \hat{ps} \rangle $ & $ \rhd $ & $ \langle c_1:c, e, ps \rangle $ &\\
$\langle \texttt{CATCH}(c_1):c, e, ps \rangle $ & $ \rhd $ & $ \langle c, e, ps \rangle $ &\\
$\langle c_1:c, e, \hat {ps} \rangle$ & $\rhd$ & $\langle c':c, e', \hat {ps} \rangle$
&\parbox{5.5cm}{ if $\langle c_1, e, ps \rangle \rhd \langle c', e', ps \rangle \text{ and} \\ c_1 \notin \{ \texttt{BRANCH}, \texttt{LOOP}, \texttt{TRY}, \texttt{CATCH} \}$}\\

\hline
\end{tabular}
\caption{Operational Semantics for abstract execution in the Sign Detection domain}
\label{table:am_rules}
\end{table}

\section{Implementation}
To support abstract domain execution the previously developed abstract machine was modified to use user-provided functions rather than
the regular arithmetic operations.
Also, \verb|Configuration.step| no longer returns a single configuration but rather a set of possible configurations because of the non-determinism introduced.

To support our rules regarding division we had to extend the supplied \verb|Operations| interface to include functions \verb|possiblyZero, isDefZero, removeZero|.
\begin{itemize}
\item
\verb|possiblyZero(A)| will return true if the arithmetic expression \verb|A| contains zero (for example \verb|possiblyZero(NON_POS)| would return true,
whereas \verb|possiblyZero(NON_ZERO)| would not).

\item
\verb|isDefZero(A)| will return true if \verb|A| is known to be exactly zero.

\item
\verb|removeZero(A)| will return the abstract value obtained by removing the abstract representation of zero from \verb|A| (for example, \verb|removeZero(NON_POS)| would return \verb|NEG|).

\end{itemize}

\subsection{Traversal}
In order to have as much precision as possible, the program is executed from every combination
of initial assignments of values (that the abstract domain provides) to variables.
Assume for example that a program has two variables $x$ and $y$.
The program will be run $3^2 = 9$ times, since there are $3$ different starting values (\verb|NEG, ZERO, POS|) for
each of the two variables.

Our implementation explores the configuration graph depth-first.
This design decision was made because it simplifies the implementation, and it becomes easier to trace execution.

To obtain the correct values for property states at each control point,
every control point is mapped to a property state, and this mapping is updated
whenever we visit this control point.
This mapping is initialized so that every control point maps to \verb|INIT|.
The updating simply computes the \emph{least upper bound} of variables in the previous mapping and the mapping of the current configuration.

Imagine a control point in our program with the $x$ and $y$ variables.
Initially it is associated with the mapping $\{x \mapsto \texttt{NONE}_A, y \mapsto \texttt{NONE}_A \}$.
Assume we reach this control point in a configuration with a property state 
$ps = \{x \mapsto \texttt{NEG}, y \mapsto \texttt{ZERO} \}$.
The mapping associated with this control point is then updated to
$\{ x \mapsto \texttt{NONE}_A \sqcup \texttt{NEG}, y \mapsto \texttt{NONE}_A \sqcup \texttt{ZERO} \} 
= \{x \mapsto \texttt{NEG}, y \mapsto \texttt{ZERO} \} $.
After every execution, the mapping at each control point will
contain the least upper bound of the possible values for each variable at that control point.

\clearpage
\section{Results}

In this section we will present some annotated programs, describe what conclusions can be drawn from the annotations,
and how this demonstrates different strengths and weaknesses of the sign analysis.

\subsection{Hello Analysis.while}

\begin{figure}[h!]
\centering
\begin{verbatimtab}[4]
{y=Z, x=Z}
try
    {y=Z, x=Z} rhs: T
    if y <= 0 then
        {y=NON_POS, x=Z} rhs: NON_NEG
        y := (0 - y)
    else
        skip;
    {y=NON_NEG, x=Z} rhs: ANY_A
    x := (5 / y)
catch
    {y=ZERO, x=Z} rhs: POS
    x := 3;
{y=NON_NEG, x=NON_NEG}
Abnormal termination impossible!
Normal termination possible!
\end{verbatimtab}

\label{hello_analysis.while}
\caption{A simple annotated program}
\end{figure}

As can be seen by the above annotated program,
the signs analysis is able to deduce that \emph{if this program terminates},
variables $x$ and $y$ will be non-negative. 
This is as expected because the program is equivalent to $x = \lfloor 5 / |y| \rfloor, y = |y|$ if $y \neq 0$,
and $x = 3, y = |y|$ otherwise.

This analysis is possible, because we iterate over every possible start configuration,
rather than starting from assigning every variable to $Z$.

\clearpage
\subsection{Division Precision.while}
\begin{figure}[h!]
\centering
\begin{verbatimtab}[4]
{z=Z, y=Z, x=Z} rhs: POS
x := 5;
{z=Z, y=Z, x=POS} rhs: POS
y := 3;
{z=Z, y=POS, x=POS}
try
    {z=Z, y=POS, x=POS} rhs: NON_NEG
    z := (x / y);
    {z=NON_NEG, y=POS, x=POS} rhs: ANY_A
    x := (x / z)
catch
    skip;
{z=NON_NEG, y=POS, x=NON_NEG} rhs: POS
y := (x + 2);
{z=NON_NEG, y=POS, x=NON_NEG}
Abnormal termination impossible!
Normal termination possible!
\end{verbatimtab}

\label{division_precision.while}
\caption{Example program analysis which is possible because of the intact precision of the division}
\end{figure}

The above example shows a case where loss of precision is avoided in the division operation.
This is because the divisor is a \verb|NON_NEG|, so we branch into two possible futures:
$\frac{x}{\texttt{ZERO}}$ and $\frac{x}{\texttt{POS}}$.
The analysis would have been less precise if the second branch had executed $\frac{x}{\texttt{Z}}$ instead.
$y$ is correctly deduced to be \verb|POS| after this program, which would have been impossible with the less precise division.

\clearpage
\subsection{Else What?.while}

\begin{figure}[h!]
\centering
\begin{verbatimtab}[4]
{y=Z, x=Z}
try
    {y=Z, x=Z} rhs: ANY_A
    x := (5 / y);
    {y=NON_ZERO, x=Z} rhs: FF
    if y = 0 then
        {y=NONE_A, x=NONE_A} rhs: NONE (Dead Code)
        x := 3
    else
        skip
catch
    {y=ZERO, x=Z} rhs: POS
    x := 45;
{y=Z, x=Z}
Abnormal termination impossible!
Normal termination possible!
\end{verbatimtab}

\label{waste_if.while}
\caption{Example program analysis which shows detection of dead code (unnecessary if-then-else)}
\end{figure}

The above example shows that it is possible for the analysis to detect some dead code in an if-statement.
This is possible because all code is initially marked as dead, until it is visited for the first time.
Note that the branch where $y=\texttt{ZERO}$ does not mark the then clause as visited,
even though it visits that instruction.
This is because it has actually aborted at the division, and is simply ignoring instructions up until the next catch (or end of program).

This kind of analysis supports program transformation that removes unnecessary if then else statements.

\clearpage
\subsection{Wasted Catch}
\begin{figure}[h!]
\centering
\begin{verbatimtab}[4]
{z=Z, y=Z, x=Z} rhs: ANY_A
z := (5 / z);
{z=Z, y=Z, x=Z}
try
    {z=Z, y=Z, x=Z} rhs: T
    if !(x = 0) then
        {z=Z, y=Z, x=NON_ZERO} rhs: Z
        y := (100 / x)
    else
        {z=Z, y=Z, x=ZERO} rhs: POS
        y := 100
catch
    {z=NONE_A, y=NONE_A, x=NONE_A} rhs: NONE (Dead Code)
    x := 5;
{z=Z, y=Z, x=Z}
Abnormal termination possible!
Normal termination possible!
\end{verbatimtab}
\label{waste_catch.while}
\caption{Example program analysis which shows detection of dead code (unnecessary try-catch) and uncaught exception}
\end{figure}

This example program shows that the analysis is able to detect:

\begin{itemize}
\item	That the try-catch is unnecessary. This is because the division cannot be executed with $x = 0$ due to the if-guard.
\item	That the program may terminate with an uncaught exception (because of the first division assignment).
\end{itemize}

\clearpage
\subsection{Analyze this.while}
\begin{figure}[h!]
\centering
\begin{verbatimtab}[4]
{z=Z} rhs: T
if 5 <= 3 then
    {z=Z} rhs: POS
    z := 5
else
    {z=Z} rhs: NEG
    z := (0 - 5);
{z=NON_ZERO}
Abnormal termination impossible!
Normal termination possible!
\end{verbatimtab}
\label{waste_catch.while}
\caption{Example program analysis which shows limitation of the signs analysis}
\end{figure}

This example program shows that the analysis cannot deduce things about some constant expressions.
This is simply because of the lack of granularity in the domain.
This sample program would obviously get a better analysis from the constant propagation domain.


\end{document}
